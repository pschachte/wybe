%  Emacs    : -*- latex -*-
%  File     : design.tex
%  RCS      : $Id: design.tex,v 1.14 2008/11/19 22:50:59 schachte Exp $
%  Author   : Peter Schachte
%  Origin   : Fri Mar 28 19:36:48 2003
%  Purpose  : Discussion document on design of new language
%  Copyright: � 2003 Peter Schachte.  All rights reserved.
%

\documentclass{article}
\usepackage{xspace}
%\usepackage{ensuremath}

\newcommand{\lang}{\textsc{Wybe}\xspace}
\newcommand{\Lang}{\textsc{Wybe}\xspace}

\newcommand{\ie}{\emph{i.e.}\xspace}
\newcommand{\eg}{\emph{e.g.}\xspace}
\newcommand{\Ie}{\emph{I.e.}\xspace}
\newcommand{\Eg}{\emph{E.g.}\xspace}
\newcommand{\etc}{\emph{etc}}

\newcommand{\tuple}[1]{\ensuremath{\langle#1\rangle}}

\title{The \Lang Programming Language}
\author{Peter Schachte}

\begin{document}
\maketitle

\section{Aims}

The aim of this project is to produce a clean, easy-to-learn,
easy-to-use programming language suitable for large teams to produce
maintainable programs.  It should be suitable as a first programming
language, and should be efficient enough for serious use.


\section{Modules}

Each source file is a module, and may contain any number of the
following things, in any order:
\begin{itemize}
\item Type definitions
\item (Sub-)modules
\item Module imports
\item Function definitions
\item Procedure definitions
\item Statements
\end{itemize}

\subsection{Imports}
\Lang supports the following varieties of module importation:
\begin{itemize}
\item Members may be imported such that they may be used as if they
  were members of the importing module (the \texttt{import}
  directive), or may be imported such that they must be specified with
  fully qualified names (the \texttt{use} directive).
\item A full module may be imported (\texttt{import} or \texttt{use}
  \emph{module}), or only certain specified members
  may be imported (\texttt{from} \emph{module} \texttt{import}
  \emph{items} or \texttt{from} \emph{module} \texttt{use}
  \emph{items}).
\item Members may be imported such that they become public members of
  the \emph{importing} module, visible to any module that imports it
  (by preceding the \texttt{import} or \texttt{use} directive with
  \texttt{public}) or not.
\end{itemize}
This gives the following eight varieties of importation:
\begin{center}
  \begin{tabular}{|p{2.4cm}|p{8.5cm}|}
    \hline
    \texttt{import} \emph{module}
    & Imports everything from \emph{module} so that it can be used
    with or without module qualification.
    \\ \hline
    \texttt{use} \emph{module}
    & Imports everything from \emph{module} so that it can only be used
    with module qualification.
    \\ \hline
    \texttt{from} \emph{module} \texttt{import} \emph{items}
    & Imports the specified items from \emph{module} so that they
    can be used with or without module qualification.
    \\ \hline
    \texttt{from} \emph{module} \texttt{use} \emph{items}
    & Imports the specified items from \emph{module} so that they
    can only be used with module qualification.
    \\ \hline
    \texttt{public import} \emph{module}
    & Imports everything from \emph{module} so that it can be used
    with or without module qualification, and reexport it from the
    importing module as if it were defined there.
    \\ \hline
    \texttt{public use} \emph{module}
    & Imports everything from \emph{module} so that it can only be used
    with module qualification, and reexport it as a member of a submodule.
    \\ \hline
    \texttt{public from} \emph{module} \texttt{import} \emph{items}
    & Imports the specified items from \emph{module} so that they
    can be used with or without module qualification, and reexport
    them from the importing module as if they were defined there.
    \\ \hline
    \texttt{public from} \emph{module} \texttt{use} \emph{items}
    & Imports the specified items from \emph{module} so that they
    can only be used with module qualification, and reexport them as
    members of a submodule.
    \\ \hline
  \end{tabular}
\end{center}

\section{Types}

The language is strongly typed, with parametric polymorphism
and type inference.

All data types are effectively abstract.  That means
constructors (and deconstructors) are ordinary functions.  Thus an
existing type with defined constructors and deconstructors can be
redefined with new functions implementing the previous constructors
and deconstructors.

A type can then be viewed as just the set of functions and predicates that
operate on it.  That is, a type is just a special kind of module that
provides at least one constructor.


\section{Explicit dataflow indicators}


\end{document}

% LocalWords:  vararg predicate's nonfailure DCG monoporphized nondet RDBMS
% LocalWords:  coroutining distfix ness supertyping Liskov backquote
% LocalWords:  restrictable
